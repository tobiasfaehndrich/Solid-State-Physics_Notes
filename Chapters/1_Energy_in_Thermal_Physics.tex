\def \secname {Energy in Thermal Physics}

\section[\secname]{\hyperlink{toc}{\secname}}



\subsection{Thermal Equilibrium}
\textbf{Temperature} (operational definition): What you measure with a thermometer

\textbf{Temperature} (theoretical definition): The thing that is the same for two objects, after they've been in contact long enough

\textbf{Thermal Equilibrium}: 2 objects in contact for long enough

\textbf{Relaxation Time}: Time taken to get to thermal equilibrium. (technically time to drop by a factor if e)

\textbf{Diffusive Equilibrium}: ex. cream and coffee mixing; exchange of \textbf{particles}

\textbf{Mechanical Equilibrium}: ex. expansion of balloon; exchange of \textbf{volume}

\subsection{The Ideal Gas}
\begin{shaded}
\textbf{The Famous Ideal Gas Law} \newline
\begin{equation}
PV = nRT
\end{equation}
Where:
\begin{equation*}
\begin{split}
P = \text{Pressure} \\
V = \text{Volume} \\
n = \text{number of moles of the gas}\\
R = \text{Universal Constant; 8.31}\frac{J}{mol K}\\
T = \text{Temperature in Kelvins}
\end{split}
\end{equation*}
\end{shaded}

\textbf{Conversions and Constants}:

\[\text{Pa (pascals, SI base units for pressure)} = \frac{N}{m^2}\]
\[1 \text{atm} = 1.013 \text{bar} = 1.013 \times 10^{5} Pa\]
\[1 \text{bar} = 10^5Pa\]
\[1 L = 0.001m^3\]
\[1 mL = 1cm^3\]
\[ 1eV = 1.602 \times 10^{-19}J\]
\[ (\text{T in }^\circ C) = (\text{T in K}) - 273.15\]
\[ (\text{T in }^\circ F) = \frac{9}{5}(\text{T in }^\circ C) +32\]
\[1 \text{cal} = 4.186J\]
\[1 u = 1.661 \times 10^{-27} kg\]

\textbf{Boltzmann Constant:}
\[k= 1.3806488 \times 10^{-23} \frac{J}{K}\ = 8.62 \times 10^{-5}\frac{eV}{K}\]
\textbf{Avogadro's Number:}
\[N_A = 6.02 \times 10^{23}\]
\textbf{Gas Universal Constant:}
\[R = k N_A = 8.3144621 \frac{J}{mol K}\]

\textbf{Alternate Ideal Gas Law:}
\[PV = NkT\]

\[nR = Nk\]


\subsection{The Equipartition of Energy}
\begin{shaded}
\textbf{Equipartition Theorem}\newline 
At temp T, the average energy of an quadratic degree of freedom is $\frac{1}{2}kT$

\begin{equation}
U_{thermal} = Nf \frac{1}{2} kT
\end{equation}
Where:
\begin{equation*}
\begin{split}
N = \text{Number of molecules} \\
f = \text{Number of degrees of freedom} \\
k = \text{Boltzmann Constant}\\
T = \text{Temperature in Kelvins}
\end{split}
\end{equation*}
\end{shaded}
\begin{note}

\textbf{$N_2$ and $O_2$ have f=5, Gas of Monatomic Molecules like Helium or Argon f=3, Solids have f=6}
\end{note}


\subsection{Heat and Work}
\textbf{Heat} - Spontaneous Flow of energy from one object to another, caused by difference in temperature between objects \newline
\textbf{Work} - Any other transfer of energy. Usually can identify external agent that actively puts energy into system.\newline
\textbf{Energy} - Most fundamental dynamical concept in physics. Cannot be created or destroyed.\newline
\textbf{Temperature} - Fundamentally an objects tendency to spontaneously give up energy \newline

\begin{shaded}
\textbf{The First Law of Thermodynamics}\newline 
Essentially Conservation of Energy

\begin{equation}
dU = Q + W
\end{equation}
Where:
\begin{equation*}
\begin{split}
U = \text{total energy inside a system} \\
Q = \text{amount of energy that enters system as heat} \\
W = \text{amount of energy that enters system as work}\\
\end{split}
\end{equation*}
\end{shaded}


\textbf{Heat Transfer Categories:} \\
\textbf{Conduction}: molecular contact/bumping and transferring energy \\
\textbf{Convection}: bulk motion of gas/liquid (usually warmer rises) \\
\textbf{Radiation}: emission of EM waves, mostly infrared for objects at room temp but also visible for hot objects like lightbulbs.


\subsection{Compression Work}
\textbf{Quasistatic} - if piston is compressed slowly, gas has time to continually equilibrate to changing conditions (never perfectly).

\begin{equation}
W = -P \Delta V
\end{equation}

If pressure changes as you change the volume (still quasistatic):

\begin{equation}
    W = - \int_{V_i}^{V_f} P(V) dV
\end{equation}

\textbf{Isothermal Compression} - Compress gas slow that the temperature of th egas doesn't rise at all.
\textbf{Adiabatic Compression} - Compress gas so fast that no heat escapes from gas during process.

\subsection{Heat Capacities}

\textbf{Heat Capacity} - The amount of heat needed to raise it's temperature, per degree temperature increase.
\[C \equiv \frac{Q}{\Delta T} = \frac{\Delta U - W}{\Delta T}\]

\textbf{Specific Heat Capacity} - Heat capacity per unit mass.
\[c \equiv \frac{C}{m}\]
\begin{shaded}
\textbf{Heat Capacity at Constant Volume}
\begin{equation}
C_V = \left(\frac{\Delta U}{\Delta T}\right)_V = \left(\frac{\partial U}{\partial T}\right)_V
\end{equation}
\end{shaded}

\textbf{Heat Capacity at Constant Pressure}
\begin{equation}
C_P = \left(\frac{\Delta U - (-P\Delta V)}{\Delta T}\right)_P = \left(\frac{\partial U}{\partial T}\right)_P + P\left(\frac{\partial V}{\partial T}\right)_P
\end{equation}
\textbf{Recall}
Section 1.3 - Equipartition Theorem: $U_{thermal} = Nf \frac{1}{2} kT$
\begin{note}
$C_V$ for monatomic gas (f=3) (ex Helium) is $\frac{3}{2}Nk=\frac{3}{2}nR$. 

\hspace{15mm} \textbf{Rule of Dulong and Petit} Heat capacity per mole for solid (f=6) is $3Nk=3nR$. 
\end{note}

\begin{shaded}
\textbf{Heat capacity at constant pressure for Ideal Gas}
\begin{equation}
    C_P = C_V + Nk = C_V + nR
\end{equation}
\end{shaded}

\textbf{Latent Heat}
\textbf{phase transformation} - ex. melting ice or boiling water

At phase transformation you can put heat into system without increasing the temperature.
\[C=\frac{Q}{\Delta T} = \frac{Q}{0} = \infty\]
Heat capacity is infinite during a phase transformation

\textbf{Latent Heat} - How much heat is required for transformation (ex. melting water)
\[ L \equiv \frac{Q}{m} \]
\begin{note}
By convention we assume pressure is constant (usually at 1 atm) and that no other work is being done besides the usual constant-pressure expansion/compression.
\end{note}
\textbf{Reference}:\\
L for melting ice $ = 333\frac{J}{g}  = 80\frac{\text{cal}}{g}$ \\
L for boiling water $ = 2260\frac{J}{g}  = 540\frac{\text{cal}}{g}$ \\
Raising Water 0 to 100 requires only $100\frac{\text{cal}}{g}$

\begin{shaded}
\textbf{Enthalpy}
\[H \equiv U + PV\]
\begin{equation}
    \Delta H = Q + W_{other} \text{ \hspace{10mm} ; constant P}
\end{equation}
\end{shaded}

$\therefore$ for simple case of raising object's temperature:
\[C_P = \left(\frac{\partial H}{\partial T}\right)_P \]

\subsection{Rates of Processes}
Not entirely thermodynamics, but related. "Transport Theory or Kinetics"\\
Review this Chapter only if required