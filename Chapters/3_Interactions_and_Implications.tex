\def \secname {}

\section[\secname]{\hyperlink{toc}{\secname}}




\subsection{Temperature}

\textbf{At equilibrium}
\begin{equation}
\frac{\partial S_A}{\partial U_A} = \frac{\partial S_B}{\partial U_B}
\end{equation}

The slope of their entropy vs energy graphs are the same for both systems when they're in thermal equilibrium. This must be related to the temperature of a system.

Energy will tend to flow \textbf{into} the object with the \textbf{steeper} S vs U graph, and \textbf{out} of the object with the \textbf{shallower} S vs U graph.

\textbf{Steep Slope -- Low Temperature}

\textbf{Shallow Slope -- High Temperature}

$\therefore$ The Temperature of a system is the reciprocal of the slope of it's entropy vs energy graph. The partial derivative is to be taken with the systems volume and number of particles fixed.
\begin{shaded}
New Definition of Temperature:
\begin{equation}
    T \equiv \left( \frac{\partial S}{\partial U} \right)^{-1} \text{   or   } \frac{1}{T} = \left(\frac{\partial S}{\partial U}\right)_{U, V}
\end{equation}
\end{shaded}

\textbf{Silly Analogy}: money is energy, happiness is entropy, and generosity is temperature.

\subsection{Entropy and Heat}

\textbf{Steps to Predicting Heat Capacities}

1. Use quantum mechanics and some combinatorics to find an expression for the multiplicity, $\Omega$, in terms of U, V, N and any other relevant variables.\\
2. Take the logarithm to find entropy, S.\\
3. Differentiate S with respect to U and take the Reciprocal to find the temperature, T, as a function of U and other variables.\\
4. Solve for U as a function of T (and other variables)\\
5. Differentiate U(t) to obtain a prediction for the heat capacity (with the other variables held fixed).

(main examples: two-state paramagnet, Einstein Solid, and the monatomic ideal gas.)

\textbf{Einstein Solid}; $q \gg N$; $U=NkT$
\[C_V = \frac{\partial}{\partial T}(NkT) = Nk\]

\textbf{Monatomic Ideal Gas}; $U=\frac{3}{2}NkT$
\[C_V = \frac{\partial}{\partial T}(\frac{3}{2}NkT) = \frac{3}{2}Nk\]

\textbf{Measuring Entropies}

from $\frac{dS}{dU}=\frac{1}{T}$ and $dU = Q + \cancel{W} $ we see that:

\begin{equation}
    dS = \frac{dU}{T} = \frac{Q}{T} ; \text{constant volume, no work}
\end{equation}

$dS = \frac{Q}{T}$ is also true when the volume is changing provided that the process is quasistatic.

When T is changing it is easier to write/use:
\[dS = \frac{C_V dT}{T}\]
\begin{equation}
 \Delta S = S_f - S_i = \int_{T_i}^{T_f} \frac{C_v}{T}dT
\end{equation}

"The Third Law of Thermodynamics"

To find system's total Energy, simply take $T=0$ for lower limit. In principle S(0) is 0 because it is absolute 0 Kelvin (i.e. $\Omega = 1$).

as $C_V \rightarrow 0$, $T \rightarrow 0$

\textbf{The Macroscopic View of Entropy}

If one system loses some entropy as heat flows out, another system will gain even more entropy (as heat enters into it).

\subsubsection{Paramagnetism}

We will look at the harder problem of the two-state paramagnet. Dipoles spin is quantized, either up or down. Ideal paramagnet, there is no interactions between dipoles, only effected by magnetic field.

\begin{equation}
    U_{total} = \mu B(N\downarrow - N\uparrow) = \mu B(N-2N\uparrow)
\end{equation}

\begin{equation}
    M = \mu (N\uparrow - N\downarrow) = \frac{-U}{B}
\end{equation}

\begin{equation}
    \Omega (N\uparrow) = \frac{N!}{N\uparrow!N\downarrow!}
\end{equation}

\begin{note}
lots of counter-intuitive results p.100-103
\end{note}

\textbf{Numerical Solution}

we find that multiplicity and thus entropy is \textbf{greatest} when $U = 0$; half $N\uparrow $ and half $N\downarrow$.

\textbf{Analytic Solution}
\begin{equation}
    \frac{S}{k} = N\ln(N) - N\uparrow \ln(N\uparrow) - (N-N\uparrow)\ln(N-N\uparrow)
\end{equation}

\begin{equation}
\frac{1}{T} = \frac{k}{2\mu B}\ln\left(\frac{N-U/(\mu B)}{N+U/(\mu B)}\right)  
\end{equation}

\begin{equation}
    U = N \mu B \left( \frac{1-e^{2\mu B/kT}}{1+e^{2\mu B/kT}}\right) = -N\mu B \tanh{\left(\frac{\mu B}{kT}\right)}
\end{equation}

\begin{equation}
    M = N \mu \tanh{\left(\frac{\mu B}{kT}\right)} ; \text{as T $\rightarrow$ $\infty$, M $\rightarrow$ 0}
\end{equation}


\begin{equation}
    C_B = \left(\frac{\partial U}{\partial T}\right)_{N, B} = Nk \frac{(\mu B/kT)^2}{\cosh^2{(\mu B/kT)}}
\end{equation}

\textbf{Where $\mu$ is the Bohr Magneton}
\[\mu_B \equiv \frac{eh}{4\pi m_e} = 9.274 \times 10^{-24} \frac{J}{T} = 5.788 \times 10^{-5} \frac{eV}{T}\]

e is electron charge = $1.602 \times 10^{-19}$

$m_e$ is electron mass = $9.11 \times 10^{-31}$


\textbf{Hyperbolic Trig Functions}
\begin{equation*}
\begin{split}
   \sinh{x} = \frac{1}{2}(e^x - e^{-x})\\
   \cosh{x} = \frac{1}{2}(e^x + e^{-x})\\
   \tanh{x} = \frac{\sinh{x}}{\cosh{x}}
\end{split}
\end{equation*}

\textbf{using $tanh{x} \approx x$ for $x \ll 1$ we get:}

\[M \approx \frac{N\mu ^2 B}{kT} \]

\textbf{Currie's Law (Pierre Curie)}

$M \propto \frac{1}{T}$

This holds for high temperature limit for paramagnets.

\subsection{Mechanical Equilirium and Pressure}

\textbf{Pressure}

\begin{equation}
    P = T \left(\frac{\partial S}{\partial V} \right)_{V, N}
\end{equation}


\textbf{The Thermodynamic Identity}
$dS = \frac{1}{T}dU + \frac{P}{T} dV$

(assuming number of particles is fixed)

\begin{equation}
    dU = TdS - PdV
\end{equation}

\textbf{Entropy and Heat Revisited}
\begin{note}
Recall FLTD; $dU = Q + W$, where $W= -PdV$ so we find that:
\end{note}

\begin{equation}
    Q = TdS \text{; (if quasistatic)}
\end{equation}

or $dS = \frac{Q}{T}$ ; even if work is being done on it during the process.

\textbf{Isentropic - } \textbf{adiabatic} (compress fast enough so that gas heats up) and \textbf{quasistatic}3 (compress slow enough that the particles are able to constantly equilibrate)

Watch out if system is not quasistatic. ex: very fast compression that creates internal disequilibrium (work you do on gas is greater than $-PdV$). Also free expansion into a vacuum.

\subsection{Diffusive Equilibrium and Chemical Potential}

\textbf{Thermal Equilibrium} $\rightarrow$ same temp

\textbf{Mechanical Equilibrium} $\rightarrow$ same pressure

\textbf{Chemical Potential $\mu$:}

\begin{equation}
    \mu \equiv -T \left(\frac{\partial S}{\partial N}\right)_{U,V}
\end{equation}

$\mu_A = \mu_B$ at equilibrium

$\therefore$ diffusive equilibriu, $\rightarrow$ same chemical potential

Particles tend to flow from the system with higher $\mu$ into the system with the lower $\mu$.

Adding this to the thermodynamic identity, we find:
\begin{equation}
  dU = TdS - PdV + \mu dN  
\end{equation}

Where
\begin{equation*}
\begin{split}
-PdV = \text{mechanical work}\\
\mu dN = \text{chemical work}
\end{split}
\end{equation*}

\begin{note}
If system contains several types of particles (i.e. like air) then each molecular specie has it's own chemical potential.

We also see:
\end{note}
\[ \mu = \left(\frac{\partial U}{\partial N}\right)_{S,V}\]

\textbf{$\therefore$ Generalized Thermodynamic Identity}

\begin{equation}
    dU = TdS - PdV + \sum \mu_i dN_i
\end{equation}
(Where the sum runs over all species, i =1,2,3...)

\subsection{Summary and Look Ahead}

\textbf{"Entropy tends to increase!!!!"}

\begin{tabular}{ |p{3.5cm}|p{3.5cm}|p{3.5cm}|p{3cm}|  }
 \hline
 \multicolumn{4}{|c|}{Overview} \\
 \hline
 Type of Interaction & Exchanged Quantity & Governing Variable & Formula\\
 \hline
 Thermal   & Energy    & Temperature &   $\frac{1}{T} = \left(\frac{\partial S}{\partial U}\right)_{V,N} $\\
 Mechanical &   Volume  & Pressure   & $\frac{P}{T} = \left(\frac{\partial S}{\partial V}\right)_{U,N} $\\
 Diffusive & Particles & Chemical Potential & $\frac{\mu}{T} = -\left(\frac{\partial S}{\partial N}\right)_{V,U} $\\
 \hline
\end{tabular}

\textbf{3 key model systems:}\\ - the Two State Paramagnet\\ - the Einstein Solid\\ - the Monatomic Ideal Gas.
