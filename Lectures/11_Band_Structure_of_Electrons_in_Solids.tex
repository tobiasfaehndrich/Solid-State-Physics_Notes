 \section[Band Structure of Electrons in Solids]{\hyperlink{toc}{Band Structure of Electrons in Solids}}



 \textbf{Band structure representation for 1D}
 \begin{itemize}
     \item Reduced zone scheme
     \item Extended zone scheme 
 \end{itemize}

\[E = \epsilon_0 - 2J \cos(ka) \] 

- for Both we see energy band gaps

- At low temp a half filled band involves states being filled up to the Fermi Energy

- \textbf{fermi surface} is the points or planes in k where the filled meets the unfilled region

- \textbf{fermi energy} is the line in E where the filled meets the unfilled region

- \textbf{Unfilled band} allows electric field to shift electrons in band ==> conducting

- \textbf{Filled band} has no available states to move into and thus (almost) never carries a current.

- \textbf{Band filling and Fermi surfaces in 2D}

- fills up in same way $\xrightarrow{}$ lowest energy first, each level with at most one electron in each spin state.

2D hopping/tight-binding model for cubic lattic, we get energy:
\begin{equation}
    E = \epsilon_0 - 2J\cos(k_x a) - 2J \cos(k_y a)
\end{equation}


- highest unfilled band is the conduction band

- valence band is the bands that are filled.

from $\frac{p^2}{2m}$:

\textbf{lighter effective mass} $\xrightarrow{}$ sharper parabola in band

\textbf{heavier effective mass} $\xrightarrow{}$ wider parabola in band 

\textbf{Band theory does not consider interactions between electrons!} -- cannot explain:
\begin{itemize}
    \item Magnetism
    \item Mott insulators
    \item Superconductivity
\end{itemize}

- \textbf{Band insulators}: cannot absorb photons with smaller energy than the bandgap. Instead they pass through and it is transparent to them
- if BG $>$ 3.2eV, then they are transparent with visible light (ex: Quartz, diamond, sapphire (aluminium oxide).

- \textbf{Semiconductors:} smaller bandgaps can absorb for ex. violet and blue light and transmit red and green (like CdS). 
- Other smaller ($<$1.7eV) BGs may transmit in the IR but appear black cause they absorb visible light (ex. Si, Ge, GaAs).

\begin{itemize}
    \item \textbf{Direct band gap}: VB max and CB min are at the same k value.
    \item \textbf{Indirect band gap}: VB max and CB min are not at the same k value.
\end{itemize}

-indirect band gap makes it hard for material to absorb light because you need an increase in crystal momentum (which again is hard)! Not good for optical applications for lasers (but rather for electronics like silicon).
