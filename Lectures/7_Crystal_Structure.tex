\section[Crystal Structure]{\hyperlink{toc}{Crystal Structure}}


\begin{itemize}
    \item crystals: periodic arrangements of atoms in a lattice
    \item Penrose tyles are quasi crystal.
    \item \textbf{Lattice:} repeating structure defined as an infinite set of points, which can be constructed as sums of set of linearly independent \textbf{primitive lattice vectors}.
    \[ \vec{R} = n_1 \vec{a}_1 + n_2 \vec{a}_2 + n_3 \vec{a}_3, \qquad n \in \{0, \pm1, \pm2 ...\} \]
    \item Points in the form of: $2\vec{a}_1 + \vec{a}_2$ notated as $[2,1]$
    \item \underline{Square Lattice:} $\phi = 90^{\degree}, \qquad |a_1|=|a_2|$
    \item \underline{Hexagonal Lattice} (or Triangular): $\phi = 120^{\degree}, \qquad |a_1|=|a_2|$
    \item choice of primitive lattice vectors for a lattice is not unique $\rightarrow$ must be able to get to any spot/point on the lattice with a linear combination.
    \item \textbf{Unit cell:} a region of space chosen so that you could tile/ fill all of space.
    \item \textbf{Primitive Unit Cell:} Unit cell with exactly one lattice point inside (conventional/convenient unit cell is when it is not primitive). Can also add fractions of points to add up to 1 full point for PUC (i.e. $4 * 1/4 = 1$).
    \item \textbf{Wigner-Seitz Unit Cell:} is a way to construct a primitive unit cell that is always possible.
    \begin{enumerate}
        \item Draw lines to connect a given lattice point to all nearby lattice points
        \item At the midway points, normal to these lines, draw new-lines or planes.
        \begin{itemize}
            \item (smallest volume enclosed in this way is the Wigner-Seitz primitive cell. All space may be filled by these cells.)
            \item \textbf{notice}: not all lines from the method need to be used in the end.
        \end{itemize}
    \end{enumerate}
    \item \textbf{Crystal with Multiple Atom Types:}
    \begin{itemize}
        \item Unit cell can have more than one atom
        \item periodic for each atom type or for some region in space repeated periodically.
        \item Define a basis for example for the large atom and for the small atom.
    \end{itemize}
    \item \textbf{Simple Cubic 3D:} very improbable in nature cause there is so much empty space where crystal could configure in a more efficient manner.
    \begin{itemize}
        \item can be \textbf{unit cell (a), tetragonal $(a = b \neq c)$ or orthorhombic $(a \neq b \neq c)$}
    \end{itemize}
    \item Cesium Chloride (CsCl): is simple cubic with basis Cs [0,0,0] and Cl [1/2, 1/2, 1/2]
    \item \textbf{Unit Cell of Body Centered Cubic Lattice (BCC):}
    \begin{itemize}
        \item Much more probable 
        \item Simple cubic with basis also Body centered cubic (BCC):
    \end{itemize}
    \item \textbf{Face Centered Cubic (FCC):}
    \begin{itemize}
        \item nature likes even more (probable!), for ex: NaCl Salt or C Diamond.
    \end{itemize}
    \item The full list of all possible 3D crystal types (\textbf{The 14 Bravais Lattice Types})
    \begin{itemize}
        \item \textbf{4 types of Unit Cells }
        \begin{itemize}
        \item P = primitive
        \item I = Body-centered
        \item F = Face-centered 
        \item C = side-centered
        \end{itemize}
        \item \textbf{with 7 crystal classes}
        \begin{enumerate}
            \item Cubic
            \item Tetragonal
            \item Orthorhombic
            \item Hexagonal
            \item Trigonal
            \item Monoclinic 
            \item Triclinic
        \end{enumerate}
    \end{itemize}
    \item GaAs is a zincblende FCC Basis atom1 (0,0,0) and atom2 (1/4, 1/4, 1/4).
\end{itemize}