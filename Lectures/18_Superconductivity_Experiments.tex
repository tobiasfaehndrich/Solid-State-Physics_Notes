\section[Superconductivity Experiments]{\hyperlink{toc}{Superconductivity Experiments}}

\begin{itemize}
    \item Interesting topic because we get neat phenomena like R=0, and also quantum mech application.
    \item Lecture is survey on superconductivity and why it is different 
    \item Not just band theory (fails here), need to account for interactions between electrons.
\end{itemize}

\textbf{Heat capacity of Solids}

\begin{itemize}
    \item Heat capacities at constant volume are defined as the rate of change of energy with temperature.

    \[ C = \frac{dE}{dT}\]

    \item at high temperature, $C=3Nk$, where N is the number of atoms in the solid and k is Boltzmann's constant (law of Dulong and Petit).

    \item Understood classically from the law of equipartition, which assigns a heat capacity of $k/2$ to each degree of freedom in the material.
\end{itemize}

\textbf{Temperature dependence of Resistivity}

\begin{itemize}
    \item Metal: for a sufficiently narrow range of temperature, make a linear approximation:

    \[ \rho(T) = \rho_0 [1+\alpha(T-T_0)]\]

    where $\alpha$ is the temp coefficient of resistivity, $T_0$ is a reference coefficient of resistivity, and $\rho_0$ is the resistivity at temperature $T_0$. 

    \item Standard resistivity of metal changes with temp. 
    
    \item Electrons in fermi gas occupy metal.

    \item 
\end{itemize}
