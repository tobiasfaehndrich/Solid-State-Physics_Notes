\section[Graphene and Carbon Nanotubes]{\hyperlink{toc}{Graphene and Carbon Nanotubes}}

 
 \begin{itemize}
     \item Carbon is versatile in bonding with other atoms
     \item Carbon is responsible for all organic chem
     \item Graphite (like in pencils) most closely related to Graphene
     \item 2 atom basis, FCC
     \item pencils (Graphite) are able to write because you can remove one layer (Graphene) at a time quite easily since each layer is bonded with van der Wall forces only.
     \item Graphene is single hexagonal lattice (not technically a lattice since you can't get to all points with primitive vectors) is exceptionally strong
     \item Konstantin Novoselov (Manchester) first to get graphene from scotch tape.
     \item colour change indicated that they had gotten to a monolayer
     \item solution to the graphene hamiltonian is found with tight binding mode, and my spliting the atoms up into 2 groups of triangular lattices. 
     \item \textbf{K and K'} points: where the 2 bands meet each other and touch (no energy gap)
     \item two seas of electrons but they can't get to each other (K and K' points are separate and different).
     \item semiconductor has a small bandgap, but graphene has filled valence band and empty conduction band, but the two bands are touching and thus it is considered and metal.
     \item the dispersion relation where it touches is called a Dirac Cone (ex Graphene)
     \item area of research here is called "klein tunneling"
     \item Electrons have extremely fast fermi velocity $v_F = 8E5 m/s$.
 \end{itemize}
 
 \textbf{Carbon Nanotubes!} - rolled up graphene
 
 \begin{itemize}
     \item Bonds on edge of graphene are happier if rolled around and connected to the other edge (forming a tube).
     \item can "roll" the graphene in different orientations
     \item for ex: armchair (up-flat-down), zig-zag (up-down), and chiral (various wrapping).
     \item Carbon Nanotubes (depending on the rolling vector) can "gap" Graphene (which is normally gapless.
     \item Armchair (n,n): $\theta=30$ always Metal
     \item Zig Zag (n,0): $\theta = 0$ then 2/3 are Metal and 1/3 are Semiconductor (for n multiples of 3)
     \item Chiral (n,m): $0<\theta<30$, 2/3 are meteal, and 1/3 are semi-conducting, n-m multiple of 3.
 \end{itemize}
 
 It turns out that there is a formula which tells us if the carbon nanotube will be conducting or not (i.e. if it won't hit the K point)
 
 \begin{equation}
     2n_1+n_2 = 3m \qquad m\in \{1,2,3, ...\}
 \end{equation}
